\hypertarget{pottery-design-plan}{%
\section{pottery Design Plan}\label{pottery-design-plan}}

\hypertarget{wo1-clayton-e.-williams}{%
\section{WO1 Clayton E. Williams}\label{wo1-clayton-e.-williams}}

\hypertarget{june-2023}{%
\section{June 2023}\label{june-2023}}

\hypertarget{project-structure}{%
\subsection{Project Structure}\label{project-structure}}

In the attempt to begin to build SOPs for future projects, the project
will be broken down as such:

\begin{verbatim}
.
├── data
│   └── Pottery.txt
├── doc
│   ├── design.md
│   ├── potter.1
│   ├── pottery.pdf
│   ├── writeup.adoc
│   └── writeup.pdf
├── include
├── Makefile
├── README.md
├── scratch.c
├── src
│   └── potter.c
└── test
\end{verbatim}

\texttt{data} - external files that are read as input to the program
\texttt{doc} - documentation for the project, such as design plan, man
page, writeup, and test plan \texttt{include} - custom header files for
various C source code files \texttt{src} - C source code files for the
project \texttt{test} - C source code files for unit testing

\hypertarget{objects-needed}{%
\subsection{Objects Needed}\label{objects-needed}}

The best data structure to use will be of a struct that includes the
size, capacity, array of student names, and array of student times to
finish the project:

\begin{Shaded}
\begin{Highlighting}[]
\KeywordTok{typedef} \KeywordTok{struct}\NormalTok{ student }\OperatorTok{\{}
    \DataTypeTok{int}\NormalTok{ size}\OperatorTok{;}
    \DataTypeTok{int}\NormalTok{ capacity}\OperatorTok{;}
    \DataTypeTok{char} \OperatorTok{**}\NormalTok{stu\_name}\OperatorTok{;}
    \DataTypeTok{int} \OperatorTok{*}\NormalTok{stu\_time}\OperatorTok{;}
\OperatorTok{\}}\NormalTok{ student}\OperatorTok{;}
\end{Highlighting}
\end{Shaded}

This provides the ability to pass the struct pointer to various
functions and easily maintain its current size and determine when it
needs to be resized.

\hypertarget{functions-needed}{%
\subsection{Functions Needed}\label{functions-needed}}

The basic functions needed will be:

\begin{Shaded}
\begin{Highlighting}[]
\NormalTok{get\_file\_size}\OperatorTok{();}
\NormalTok{clean}\OperatorTok{();}
\NormalTok{create\_struct}\OperatorTok{();}
\NormalTok{reallocate\_struct}\OperatorTok{();}
\NormalTok{get\_user\_input}\OperatorTok{();}
\NormalTok{create\_student}\OperatorTok{();}
\NormalTok{grade\_students}\OperatorTok{();}
\NormalTok{print\_students}\OperatorTok{();}
\end{Highlighting}
\end{Shaded}

\texttt{get\_file\_size} - Identifies number of lines in the data file
to allocate arrays to exact length. \texttt{clean} - run at the end of
program or at exit to free allocated memory. \texttt{create\_struct} -
initializes struct. \texttt{reallocate\_struct} - resize struct.
\texttt{get\_user\_input} - take user input from command line for manual
student entry. \texttt{create\_student} - adds student name and time to
the respective arrays. \texttt{grade\_students} - iterate over students
to identify optimal index to begin grading. \texttt{print\_students} -
print the students who failed based on index returned from
\texttt{grade\_students}.

\hypertarget{project-flow}{%
\subsection{Project Flow}\label{project-flow}}

\begin{enumerate}
\def\labelenumi{\arabic{enumi}.}
\tightlist
\item
  Define struct and write function prototypes
\item
  Define main as \texttt{int\ main\ (int\ argc,\ char*\ argv{[}{]})} and
  use argc to determine if a file name was passed to the command line.
\item
  If a file name is provided, attempt to open (exit if error). Invoke
  \texttt{get\_file\_size()} to determine number of lines in the
  program, to malloc student member variables of that size.
\item
  If no file name is provided, invoke \texttt{get\_user\_input()} that
  will take user input, line by line.
\item
  Whether student data is read line by line from command line or
  provided file name, they are both treated as char pointers taht are
  passed to \texttt{create\_student()}.
\item
  \texttt{create\_student()} will have logic to call
  \texttt{reallocate\_struct()} to increase the size of the struct
  arrays based on number of students entered from command line.
  \texttt{create\_students()} will parse the string and subsequently add
  to the struct.
\item
  When the file or user input is exhausted, the struct is fd to
  \texttt{grade\_students()}. The general logical flow is to iterate
  over from 0 to size, and determine if a students time is \textgreater=
  size * 5. After that, the loop needs to run from 1 to size, and 0 to
  1, and so on. Each call of the loop will increment a counter of
  students failed if conditional is met and compare to subsequent loops
  to find the iteration with the most failed students and save that
  index.
\item
  After grading is complete, the index with most failed is passed to
  \texttt{print\_students()} that will iterate in a similar manner, this
  time, printing the students who failed.
\item
  After this is complete, invoke the \texttt{clean()} function to free
  all allocated heap memory.
\end{enumerate}
